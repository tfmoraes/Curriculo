\cvsection{Experiência Profissional}
\begin{cventries}

  \cventry
    {Engenheiro de Software}
    {593iCAN Soluções em Impressão 3D}
    {Campinas, São Paulo}
    {Outubro de 2022 - Fevereiro de 2025}
    {
      \begin{cvitems}
      \item {Planejamento, desenvolvimento e manutenção do software de geração de geometrias e código-G para biofármacos BioScaffolds}
      \item {Python, Numpy, Signed distance functions (SDF), PySide 6 (Qt), Qt3D, Código-G}
      \end{cvitems}
    }

  \cventry
  {Mentor InVesalius}
  {Google Summer of Code}
  {Remoto}
  {2023 e 2024}
  {
    \begin{cvitems}
    \item{Projetos orientados}
      \begin{itemize}
      \item{Adicionar PACS}
      \item{Adicionar informações de tipo para funções, metodos e classes}
      \item{Edição de máscara 3D}
      \item{Modo cinemático}
      \end{itemize}
    \end{cvitems}
  }

  \cventry
    {Bolsista Pesquisador}
    {Centro de Tecnologia da Informação Renato Archer – CTI}
    {Campinas, São Paulo}
    {Março de 2016 - Abril 2022}
    {
      \begin{cvitems}
      \item {Planejamento, desenvolvimento e manutenção do software livre de imagens médicas InVesalius 3}
      \item {Planejamento, desenvolvimento e manutenção do PromedWeb (software de gerenciamento do ProMED)}
      \item {Desenvolvimento e treinamento de redes neurais \textit{Deep Learning} para segmentação de partes anatômicas}
      \item {Coorientação de alunos de iniciação científica e bolsistas PCI}
      \item {Escrita de artigos}
      \item {Python, Numpy, Scipy, VTK e GDCM}
      \item {Pytorch, Keras e Horovod}
      \item {Django, Flask, HTML, Javascript, CSS e PostgreSQL}
      \item {C, C++, Cython e Cmake}
      \end{cvitems}
    }

  \cventry
  {Analista de Sistemas}
  {FacTI - Fundação de Apoio à Capacitação em TI}
  {Campinas, São Paulo}
  {Maio de 2010 - Janeiro de 2016}
  {
    \begin{cvitems}
      \item {Planejamento, desenvolvimento e manutenção do software livre de imagens médicas InVesalius 3}
      \item {Desenvolvimento de ferramentas internas para o projeto ProMED}
      \item {Coorientação de alunos de iniciação científica e bolsistas PCI}
      \item {Escrita de artigos}
      \item {Python, Numpy, Scipy, VTK e GDCM}
      \item {Django, Flask, HTML, Javascript e CSS}
      \item {C, C++, Cython e Cmake}
      \end{cvitems}
    }

    \cventry
    {Bolsista CNPq DTI}
    {Centro de Tecnologia da Informação Renato Archer – CTI}
    {Campinas, São Paulo}
    {Agosto de 2008 - Maio de 2010}
    {
      \begin{cvitems}
      \item {Planejamento, desenvolvimento e manutenção do software livre de imagens médicas InVesalius 3}
      \item {Escrita de artigos}
      \item {Python, Numpy, Scipy, VTK e GDCM}
      \end{cvitems}
    }

    \cventry
    {Programador}
    {Tauga Soluções Informática LTDA. EPP.}
    {Sorocaba, São Paulo}
    {Junho de 2008 - Agosto de 2008}
    {
      \begin{cvitems}
      \item{Desenvolvimento do sistema comercial Aquarius, produto destinado à empresas distribuidoras de combustíveis e produtos químicos}
      \item {Experiência com o ambiente de desenvolvimento Borland Delphi 7 e com o sistema gerenciador de banco de dados Firebird}
      \end{cvitems}
    }

  \end{cventries}


\cvsection{Experiência Internacional}
\begin{cventries}

  \cventry
    {Pesquisador visitante}
    {Instituto Politécnico de Leiria (IPL)}
    {Leiria, Portugal}
    {Fevereiro de 2014}
    {Projeto Europeu International Research Exchange for Biomedical Devices Design and Prototyping - IREBID}
  \end{cventries}
